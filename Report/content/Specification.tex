\section{Specification}
\label{sec:specification}

\subsection{System-Wide Specification}
\label{sec:systemWideSpecification}
The system should run on $\pm$ 5 V as the robot will likely have a central PSU that provides this configuration.
As stated in the specification, the maximum measurable distance $d_{max}$ should be 5 m. 
The distance measurement is converted into an 8-Bit word so the minimum measurable distance $d_{min}$ should be given by Equation \eqref{eqn:minimumMeasurableDistance}
\
\begin{align}
    \begin{split}
        d_{min} & =  \frac{5\ m}{2^8-1}\\
                & =  \frac{5\ m}{255}\\
                & =   2.0\ cm
    \end{split}
    \label{eqn:minimumMeasurableDistance}
\end{align}

To get accurate results, the triangle waves measured by the differential amplifiers must have a phase difference of no more than \(\pi\) radians at the maximum measurable value of 5 m.
To reduce the time spent in the latched state (where the two signals are moving in opposite directions) the phase difference at maximum range will be reduced to \(\frac{2}{3}\pi\). 
With the speed of sound at sea level under normal conditions $V_{s}$ equal to 344 m/s, the total propagation delay $t_{p}$ imposed on the reflected signal given by Equation \eqref{eqn:propagationDelay}.
This is \(\frac{1}{3}\) of the total triangle wave period $T$. 
This gives T as 87.3 ms and a frequency F of 11.5 Hz.

\begin{align}
    \begin{split}
        t_{p} & = 2\frac{d_{p}}{V_{s}}\\
              & = 2\frac{5\ m}{344\ m/s}\\
              & = 29.1\ ms
    \end{split}
    \label{eqn:propagationDelay}
\end{align}

A maximum frequency variation $\Delta f$ of 4 kHz, resulting in a bandwidth of 8 kHz, was chosen as a compromise between simpler filter/amplifier design requiring high bandwidth, and the use of more cost-effective VCO and PLL elements requiring lower bandwidth.  

\subsection{Modulation Section Specification}

\subsubsection{Triangle Wave Generator}
\label{sec:specificationTriangleWaveGenerator}
\begin{itemize}
    \item \textbf{Frequency} - 11.5 Hz $\pm$ 1 Hz to match the system spec in Section \ref{sec:systemWideSpecification}. The theoretical maximum of this value is 17.2 Hz, beyond this value the two triangle waves will be more than $\pi$ radians out of phase when measuring 5 m and the digital output would be in the lathed state 50\% of the time.
    \item \textbf{Amplitude} - 1 V $\pm$ 0.05 V as a compromise between low power operation and low $K_{VCO}$.
    \item \textbf{Offset} - 0 V $\pm$ 0.1 V to ensure center frequency at 40 kHz. 
    The tolerance of the offset and amplitude values are to limit variations in the bandwidth of the FM signal.
\end{itemize}

\subsubsection{VCO}
\label{sec:specificationVCO}
\begin{itemize}
    \item \textbf{Center Frequency} 40 kHz $\pm$ 200 Hz. Low tolerance offset error on PLL output is minimised
    \item \textbf{$K_{VCO}$} - KVCO of 4 kHz/V $\pm$ 500 Hz as calculated in Equation \eqref{eqn:KVCO}.

    \begin{equation}
        \begin{split}
             K_{VCO} &= \frac{f_{max} - f_{min}}{V_{max} - V_{min}}\\
                     &= \frac{44 kHz - 36 kHz}{1V - -1V} = 4 kHz/V
        \end{split}
        \label{eqn:KVCO}
    \end{equation}
\end{itemize}

\subsubsection{Transmission Amplifier}
\label{sec:specificationTransmissionAmplifier}
A MOSFET should be used to amplify the square wave produced by the VCO.
An engineer should be able to set the fabricated PCB into two configurations:
\begin{enumerate}
    \item \textbf{VCO $\rightarrow$ Transducer} - The output of the VCO is fed into the positive terminal of the transducer.
    The negative terminal of the transducer is set to ground. 
    \item \textbf{VCO $\rightarrow$ MOSFET Gate} - The output of the VCO is fed into the Gate of an N-channel MOSFET. The positive terminal of the transducer is connected to 5V or an external voltage supply.
    The negative terminal of the transducer is connected to the drain of the MOSFET.
\end{enumerate}
The MOSFET should have the following parameters:
\begin{itemize}
    \item \textbf{Voltage Rating} - 20 V, higher voltages could prove dangerous for engineers working with the hardware.
    \item \textbf{Current Rating} - 1A, electroacoustic transducers are capacitive devices so current requirements are relatively low. 
\end{itemize}

\subsection{Demodulator Specification}

\subsubsection{Pre-amp}
\label{sec:specificationPreAmp}
As nothing is known about the transducers to be used on the system, few of the amplifier parameters can be specified.
However, the following criteria can be specified:
\begin{itemize}
    \item \textbf{Gain Selection} - The gain of the amplifier should easy to set using commonly available passive components
    \item \textbf{Gain Range} - The gain should be selectable between a factor of 2 and a factor of 2000.
    \item \textbf{Bandwidth Stability} - Setting the gain between 2 and 200 should not reduce the gain in the range of 3500 kHz to 45 kHz by more than 3 dB.
\end{itemize} 

\subsubsection{PLL}
\label{sec:specificationPLL}
\begin{itemize}
    \item \textbf{Centre Frequency} - 40 kHz $\pm$ 200 Hz to match that of the VCO.
    \item \textbf{Triangle Wave Output Offset} - 0V $\pm$ 0.1V to minimise systematic error at the differential amplifier.
    \item \textbf{Amplitude} 1 V $\pm$ 0.1 V to match that of the triangle wave generator.
    If necessary, this should be achieved with an additional amplifier. 
\end{itemize}

\subsection{Analogue to Digital Conversion}

\subsubsection{Differential Amplifier}
\label{sec:specificationDifferentialAmp}
Parameters for the differential amplifier rely on the parameters of the ADC so they shall be described as such. 
\begin{itemize}
    \item \textbf{Differential Gain} - A potential difference of 0.33 V, corresponding to the maximum phase difference between the triangle waves of $\frac{2}{3}\pi$ radians should result in the output of the ADC reaching the upper limit of the ADC input range $V_{ADC} \pm \frac{1}{2^{8}-1}V_{ADC}$.
    This is so the phase difference caused by the transmitted wave reflecting off an object 5 m away generates the maximum value on the ADC.
    
    \item \textbf{Rectified Output} - The polarity of the inputs will periodically switch as the triangle waves change direction.
    The output of the amplifier should include a full-wave rectifier with a lower limit below $\frac{1}{2^{8}-1}V_{ADC}$ so the minimum distance can still be measured.
\end{itemize}

\subsubsection{Differentiator}
\label{sec:specificationDifferentiator}
\begin{itemize}
    \item \textbf{Single Supply} - The output of the differentiators are used as the inputs to digital circuits so their output should not go below 0 V.

    \item \textbf{Gain} - The required gain can be calculated by first finding the minimum rate of change of the triangle wave and finding the minimum multiplier that results in an output meeting the TTL logic HIGH/LOW levels.
    The rate of change is calculated in Equation \ref{eqn:triangleWaveRateOfChange}.
    \begin{equation}
        \begin{split}
            Rate Of Change  &= \frac{Peak-to-Peak Voltage}{\frac{1}{2}*Maximum Period}\\
                            &= \frac{1 V - -1V}{\frac{1}{2}*95.3 ms}\\
                            &= 42.0 V/s
        \end{split}
        \label{eqn:triangleWaveRateOfChange}
    \end{equation} 

    \citeauthor{TTLLogicLevels} state in \citetitle{TTLLogicLevels} that the logic levels for TTL inputs are $V_{in} < 0.8 V$ for guaranteed logic LOW and $V_{in} > 2 V$ for logic HIGH.
    A gain of 0.5 will suffice.

\end{itemize}
