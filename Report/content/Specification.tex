\section{Specification}
\label{sec:specification}

\subsection{System-Wide Specification}
The system should run on 5 V as the robot will likely have a 5 V bus for other control logic.
As stated in the specification, the maximum measurable distance $d_{max}$ should be 5 m. 
The distance measurement is converted into an 8-Bit word so the minimum measurable distance $d_{min}$ should be given by Equation \ref{eqn:minimumMeasurableDistance}
\
\begin{align}
    \begin{split}
        d_{min} & =  \frac{5\ m}{2^8-1}\\
                & =  \frac{5\ m}{255}\\
                & =   2.0\ cm
    \end{split}
    \label{eqn:minimumMeasurableDistance}
\end{align}

To get accurate results, the triangle waves measured by the differential amplifiers must have a phase difference of no more than \(\pi\) radians at the maximum measurable value of 5 m.
To reduce the time spent in the latched state (where the two signals are moving in opposite directions) the phase difference at maximum range will be reduced to \(\frac{2}{3}\pi\). 
With the speed of sound at sea level under normal conditions $V_{s}$ equal to 344 m/s, the total propagation delay $t_{p}$ imposed on the reflected signal given by Equation \ref{eqn:propagationDelay}.
This is \(\frac{1}{3}\) of the total triangle wave period $T$. 
This gives T as 87.3 ms and a frequency F of 11.5 Hz.

\begin{align}
    \begin{split}
        t_{p} & = 2\frac{d_{p}}{V_{s}}\\
              & = 2\frac{5\ m}{344\ m/s}\\
              & = 29.1\ ms
    \end{split}
    \label{eqn:propagationDelay}
\end{align}

A maximum frequency variation $\Delta f$ of 2 kHz, resulting in a bandwidth of 4 kHz, was chosen as a compromise between simpler filter/amplifier design requiring high bandwidth, and the use of more cost-effective VCO and PLL elements requiring lower bandwidth.  
