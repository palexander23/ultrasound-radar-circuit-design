\section{System Block Diagram}
\label{sec:System Block Diagram}

Figure \ref{fig:systemBlockDiagram} shows the system block diagram that was implemented for this project. 
It has been split into subsections each serving a particular function.
The 8-Bit digital output of the system is sent to the microcontroller at the bottom right of the diagram. 

\begin{center}
    \begin{figure}[H]
        \includegraphics[width=0.9\textwidth]{../BlockDiagram/systemBlockDiagram.png}
        \caption{A figure showing the system block diagram from which the system described in this report was designed}
        \label{fig:systemBlockDiagram}
    \end{figure}
\end{center}

\subsection{Operational Principles}
A periodic signal, in this case a triangle wave, is modulated onto an ultrasonic carrier to produce an FM-modulated signal that is emitted by an ultrasonic transducer.
This signal propagates through the air at the speed of sound (344 m/s at sea level under normal conditions) and reflects off of any solid objects in its path. 
This reflection is converted back to an electrical signal by a second transducer.

A PLL with a center frequency equal to that of the FM modulator is used to demodulate the signal. 
The output of the loop filter will vary with the frequency of the incoming signal, accurately following the triangle wave generated in the Modulation stage. 
The reconstructed triangle wave is then compared with the wave generated at that instant. 
The potential difference between the signals is directly proportional to the time it took for the reconstructed signal to travel to and from the object off of which it reflected.
The distance from the transducers to the reflecting surface can be calculated from this potential difference. 

Below is a description of the function of each subsection and an outline of the function of each block from the practical system in Figure \ref{fig:systemBlockDiagram}. 

\subsection{Modulation}
The triangle wave is generated in the Triangle Wave Generator.
It is used as the input to the VCO with a center frequency set at 40KHz. 
The bandwidth of the FM signal is specified in the tuning sensitivity ($K_{VCO}$) of the VCO.
As the parameters of the transducers are not known, an optional transmission amplifier has been included.
On a PCB, a jumper could be used to bypass the transmission amplifier if the VCO proved suitably powerful. 
The PCB area for the transmission amplifier would only be populated if required.

\subsection{Demodulation}
The output of the receiving transducer is fed into a band pass amplifier with a bandwidth equal to that of the signal generated by the VCO. 
A PLL with a 40 KHz center frequency is used to recover the triangle wave.

\subsection{Analogue to Digital Conversion}
The the potential difference between the two signals is measured using a differential amplifier and the result is converted to a digital value using an 8-Bit ADC. 
The gain of the differential amplifier is such that the maximum possible potential difference measured produces the maximum value on the ADC output. 

The potential difference between the triangle waves is constant when they are both ramping in the same direction. 
However, when one changes direction the potential difference will fall to zero and rise again, producing erroneous data on the ADC output. 
To account for this, a latch is placed between the ADC output and the microcontroller. 
This latch is transparent when both signals are ramping in the same direction and holds its value when the signals are ramping in opposite directions. 
The erroneous data is not transmitted to the microcontroller. 
The directions of the triangle wave are determined by differentiators with high gains producing TTL '1' when ramping up and TTL '0' when ramping down. 
An XOR/XNOR gate is used to produce a $\overline{LatchEnable}$ / $LatchEnable$ control signal.
 
