\section{System Block Diagram}
\label{sec:System Block Diagram}

Figure \ref{fig:systemBlockDiagram} shows the system block diagram that was implemented for this project. 
It has been split into subsections each serving a particular function.
The 8-Bit digital output of the system is sent to the microcontroller at the bottom right of the diagram. 

\begin{center}
    \begin{figure}[H]
        \includegraphics[width=0.9\textwidth]{../BlockDiagram/systemBlockDiagram.png}
        \caption{A figure showing the system block diagram from which the system described in this report was designed}
        \label{fig:systemBlockDiagram}
    \end{figure}
\end{center}

\subsection{Operational Principles}
A periodic signal, in this case a triangle wave, is modulated onto an ultrasonic carrier to produce an FM-modulated signal that is emitted by an ultrasonic transducer.
This signal propagates through the air at the speed of sound (344 m/s at sea level under normal conditions) and reflects off of any solid objects in its path. 
This reflection is converted back to an electrical signal by a second transducer.

A PLL with a center frequency equal to that of the FM modulator is used to demodulate the signal. 
The output of the loop filter will vary with the frequency of the incoming signal, accurately following the triangle wave generated in the Modulation stage. 
The reconstructed triangle wave is then compared with the wave generated at that instant. 
The potential difference between the signals is directly proportional to the time it took for the reconstructed signal to travel to and from the object off of which it reflected.
The distance from the transducers to the reflecting surface can be calculated from this potential difference. 

Below is a description of the function of each subsection and an outline of the function of each block from the practical system in Figure \ref{fig:systemBlockDiagram}. 

\subsection{Modulation}
The triangle wave is generated in the Triangle Wave Generator.
It is used as the input to the VCO with a center frequency set at 40KHz. 
The bandwidth of the FM signal is specified in the tuning sensitivity ($K_{VCO}$) of the VCO.
As the parameters of the transducers are not known, an optional transmission amplifier has been included.
On a PCB, a jumper could be used to bypass the transmission amplifier if the VCO proved suitably powerful. 
The PCB area for the transmission amplifier would only be populated if required. 

