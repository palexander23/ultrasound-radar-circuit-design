\section{Implementation}
\label{sec:implementation}

\subsection{Modulation}
\subsubsection{Triangle Wave Generator}
\label{sec:implementationTriangleWaveGenerator}
The topology implemented for this subcircuit was found in \citetitle{TriangleWaveTopology}\cite{TriangleWaveTopology}. 
It uses an integrator to generate a ramp from the output of a non-inverting schmitt trigger. 
This ramp is fed into the input of the schmitt trigger. When the ramp reaches a schmitt trigger threshold, the polarity of the schmitt trigger output reverses, forcing the integrator to ramp in the opposite direction. 
High precision, rail-to-rail opamps and E96 resistors ensure the amplitude and offset tolerances were met. 

Figure \ref{fig:triangleWaveGeneratorSchematic} shows the schematic, and Figure \ref{fig:triangleWaveGeneratorTestBench} shows the testbench used to test it. 
The Spice error log shows the results of the measurement directives.
The offset is 0.003V and the peak-to-peak voltage is 2.001 V.
The steady-state frequency is measured as 11.3532 Hz.
Each of these falls within the tolerances presented in Section \ref{sec:specificationTriangleWaveGenerator}.
A cutting of the waveform produced can be found in Figure \ref{fig:triangleWaveGeneratorWaveform}. 


\begin{figure}[H]
    \centering 
    \includegraphics[width=\textwidth]{../Circuits/Images/TriangleWaveGenerator/schematic}
    \caption{A screencap of the Triangle Wave Generator Subcircuit}
    \label{fig:triangleWaveGeneratorSchematic}
\end{figure}

\begin{figure}[H]
    \centering 
    \includegraphics[width=\textwidth]{../Circuits/Images/TriangleWaveGenerator/TestBenchScreencap}
    \caption{A screencap of the Triangle Wave Generator Subcircuit}
    \label{fig:triangleWaveGeneratorTestBench}
\end{figure}

\begin{figure}[H]
    \centering 
    \includegraphics[width=\textwidth]{../Circuits/Images/TriangleWaveGenerator/OutputWaveform}
    \caption{A screencap of the Triangle Wave Generator Subcircuit}
    \label{fig:triangleWaveGeneratorWaveform}
\end{figure}

\subsubsection{VCO}
The VCO subcircuit is based on the LTC6990 VCO IC. 
$K_{VCO}$ and the centre frequency are set using E96 resistors.
The LTSpice schematic was adapted from a circuit generatd automatically on an online design tool provided by Analog Devices\footnote{\url{http://beta-tools.analog.com/timerblox/LTC6990}}.
The schematic is shown in Figure \ref{fig:VCOSchematic}.

The testbench varies the control voltage VC and takes frequency measurements at the appropriate times. 
The testbench is shown in Figure \ref{fig:VCOTestBench}.
The waveform produced is shown in Figure \ref{fig:VCOTestBenchWaveform}.

The measurements showed that the centre` frequency ($F_{Centre}$) is 39842 Hz and the Bandwidth of the FM Modulated signal is 7905.5 Hz.
This falls within the tolerances presented in Section \ref{sec:specificationVCO}.

This IC shifts the phase of the signal by $\pi$ radians, the FM signal frequency increases when VC drops and vica versa. 
The VCO will need to account for this. 

\begin{figure}[H]
    \centering 
    \includegraphics[width=\textwidth]{../Circuits/Images/VCO/Schematic}
    \caption{A screencap of the VCO Subcircuit}
    \label{fig:VCOSchematic}
\end{figure}

\begin{figure}[H]
    \centering 
    \includegraphics[width=\textwidth]{../Circuits/Images/VCO/TestBenchScreencap}
    \caption{The testbench fort the VCO Subcircuit}
    \label{fig:VCOTestBench}
\end{figure}

\begin{figure}[H]
    \centering 
    \includegraphics[width=\textwidth]{../Circuits/Images/VCO/TestBenchWaveform}
    \caption{The Waveform produced by the VCO Testbench}
    \label{fig:VCOTestBenchWaveform}
\end{figure}

\subsubsection{Transmission Amplifier}
Due to the low power requirements of the MOSFET, a SOT-23 package was chosen to conserve space and money. 
The Infineon IRLML6344\footnote{\url{https://uk.rs-online.com/web/p/mosfets/9134070/}} was used in the simulated version shown in Figure \ref{fig:TransmissionAmpSchematic} as it meets the power requirements discussed in Section \ref{sec:specificationTransmissionAmplifier} but any SOT-23 device could be used.

The testbench contains an Butterworth-Van Dyke equivalent circuit of a transducer with parameters taken from experimental data presented by \citeauthor{equivalentCircuit} in their \citeyear{equivalentCircuit} paper\cite{equivalentCircuit}.

Figure \ref{fig:TransmissionAmpTestBenchWaveform} shows the mock VCO output superimposed over the current through the resistor R1.
It shows the constant charge and discharge of the capacitor C1, demonstrating that the MOSFET is able to induce mechanical oscillation on this mock transducer.

\begin{figure}[H]
    \centering 
    \includegraphics[width=\textwidth]{../Circuits/Images/TransmissionAmp/Schematic}
    \caption{A screencap of the Transmission Amplifier Subcircuit}
    \label{fig:TransmissionAmpSchematic}
\end{figure}

\begin{figure}[H]
    \centering 
    \includegraphics[width=\textwidth]{../Circuits/Images/TransmissionAmp/TestBenchScreencap}
    \caption{The testbench for the Transmission Amplifier Subcircuit}
    \label{fig:TransmissionAmpTestBench}
\end{figure}

\begin{figure}[H]
    \centering 
    \includegraphics[width=\textwidth]{../Circuits/Images/TransmissionAmp/TestBenchWaveform}
    \caption{The Waveform produced by the Transmission Amplifier Testbench}
    \label{fig:TransmissionAmpTestBenchWaveform}
\end{figure}

\subsection{Demodulation}

\subsubsection{Pre-Amp}
The pre-amp was implemented using a non-inverting op-amp amplifier. 
This was to make setting the gain simple and to minimise costs. 
A high-speed op-amp was chosen to reproduce any of the square wave characteristics that were transmitted by the modulation stage.

Figure \ref{fig:preAmpSchematic} shows the schematic for the preamp.
The parameters for the resistances are set by the testbench.

Figure \ref{fig:preAmpTestBench} shows the test bench. 
The gain is stepped across the range given in Section \ref{sec:specificationPreAmp}.
Parameters passed to the pre-amp block for R1 and R2 are calculated as \(R1 = Rbase = 1k\Omega\) and \(R2 = Rbase \times gain\).
The input amplitude is reduced by the same factor that the gain is increased to approximate the gain of the system increasing to compensate for greater attenuation.
At each step the 3db point is calculated to give an approximation of the system bandwidth from 0 - 3db point. 
Figure \ref{fig:preAmpTestBenchOutput} shows the output of the measure directive, showing that the bandwidth requirements given in Section \ref{sec:specificationPreAmp} have been met.  

\begin{figure}[H]
    \centering 
    \includegraphics[width=\textwidth]{../Circuits/Images/Pre-Amp/Schematic}
    \caption{A screencap of the Pre-Amp Subcircuit}
    \label{fig:preAmpSchematic}
\end{figure}

\begin{figure}[H]
    \centering 
    \includegraphics[width=\textwidth]{../Circuits/Images/Pre-Amp/TestBenchScreencap}
    \caption{The testbench fort the Pre-Amp Subcircuit}
    \label{fig:preAmpTestBench}
\end{figure}

\begin{figure}[H]
    \centering 
    \includegraphics[width=\textwidth]{../Circuits/Images/Pre-Amp/TestBenchOutput}
    \caption{The Waveform produced by the Pre-Amp Testbench}
    \label{fig:preAmpTestBenchOutput}
\end{figure}

\subsubsection{PLL}
The TI CD4046B has been chosen to implement the PLL as it is cheap and does not require many external components. 
The ``CD4046B Phase-Locked Loop (Rev. A)'' application note was used to design the circuit\footnote{\url{https://www.ti.com/lit/an/scha002a/scha002a.pdf?&ts=1589738760062}}. 
It was not simulated in LTSpice as TI does not provide a spice model.

\begin{figure}[H]
    \centering 
    \includegraphics[width=\textwidth]{../Circuits/Images/PLL/PLLReproduction}
    \caption{The PLL subcircuiit reproduced from the example circuit.}
    \label{fig:PLLSubcircuit}
\end{figure}

Figure \ref{fig:PLLSubcircuit} contains a schematic for the PLL subcircuit reproduced from Section 4.1 of the application note. 
The passive component values have been replaced with those calculated to meet the specification presented in Section \ref{sec:specificationPLL}. 
The values are derived by following the process described in Section 4.1 of the application note. 
The VCO center frequency is set using R1 and C1 as derived from Figure 9(a).
The bandwidth, given by twice the capture frequency $f_{c}$ in the application note, was set to 8 kHz by setting $R_{3}$ and $C_{2}$ using the Equation \ref{eqn:pllBandwidth}.
The component values selected will enable the PLL to enter lock when the FM signal is received.

\begin{equation}
    \begin{split}
        2f_{c} &= \frac{1}{R_{3}C_{2}}\sqrt{\frac{2\pi f_{L}}{R_{3}C_{2}}}\\
        R_{3}C_{2}^{\frac{3}{2}} &= \frac{\sqrt{2\pi f_{L}}}{2f_{c}}
    \end{split}
    \label{eqn:pllBandwidth}
\end{equation}

With $f_{l} = f_{c} = \frac{1}{2}BW = 4 kHz$ and $C_{2} = 1 uF$:

\begin{equation*}
    \begin{split}
        R_{3}C_{2} &= 0.116s\\
        R_{3} &= 165 k\Omega
    \end{split}
\end{equation*}

\subsection{Analogue to Digital Conversion}

\subsubsection{Differentiators}
The topology for the differentiator was taken from Figure 4.69 in \citetitle{DifferentiatorTopology}\cite{DifferentiatorTopology}.
The values for the passive components were copied but this gave a gain of $\frac{1}{1000}$ so C1 was increased to 1 uF to produce the required response. 
The minimum output value was set to 0 V by adding a diode to the output of the differentiator. 
Since the differentiator output is a digital signal, the forward voltage of the diode has no impact.
To allow this diode to operate, a 1 M$\Omega$ resistor was added to the testbench output to simulate the low current drawn by a logic IC. 
The AD822 IC contains two opamps so both differentiators could be implemented using a single IC.

Figure \ref{fig:differentiatorSchematic} shows the schematic as described by \citeauthor{DifferentiatorTopology}\cite{DifferentiatorTopology}.
Figure \ref{fig:differentiatorTestBench} shows the test bench used to demonstrate the circuit. 
The input to the differentiator is driven by the triangle wave generator block as described in Section \ref{sec:implementationTriangleWaveGenerator}.
Figure \ref{fig:differentiatorTestBenchWaveform} shows the output of the differentiator. 

The differentiator produces a square wave swinging between 0 and 4.2 V.
The triangle wave is shown as well, demonstrating a clear digital signal is produced, with logic HIGH corresponding to the triangle wave ramping down, and logic LOW corresponding to the triangle wave ramping up. 
Inverting the inputs to an XOR/XNOR gate has no affect on the output signal so this inversion is not an issue. 

\begin{figure}[H]
    \centering 
    \includegraphics[width=\textwidth]{../Circuits/Images/Differentiator/Schematic}
    \caption{A screencap of the Pre-Amp Subcircuit}
    \label{fig:differentiatorSchematic}
\end{figure}

\begin{figure}[H]
    \centering 
    \includegraphics[width=\textwidth]{../Circuits/Images/Differentiator/TestBench}
    \caption{A screencap of the Pre-Amp Subcircuit}
    \label{fig:differentiatorTestBench}
\end{figure}

\begin{figure}[H]
    \centering 
    \includegraphics[width=\textwidth]{../Circuits/Images/Differentiator/TestBenchWaveform}
    \caption{A screencap of the Pre-Amp Subcircuit}
    \label{fig:differentiatorTestBenchWaveform}
\end{figure}

\subsubsection{Differential Amplifier}
The differential amplifier is split into three sections.
The first is a unity-gain inverting amplifier to remove the inversion imposed by the VCO.
The second is the LT6370 instrumentation amplifier for accurately converting the differential input to a single-ended output.
The third is a full-wave rectifier so the output signal corresponds to the \textit{magnitude} of the difference between the input signals, not the absolute difference.
The schematic is shown in Figure \ref{fig:differentialAmpSchematic}

In the event that the PLL output amplitude does not match that of the Triangle Wave Generator, the gain of the inverting amplifier can be adjusted with $R_{1}$ and $R_{2}$ so both signals are precisely matched.
The topology for the rectifier was reproduced from Figure 4.63 in \citeauthor{rectifierTopology}'s \citetitle{rectifierTopology}\cite{rectifierTopology}.
The feedback circuits of the op-amps eliminate the voltage drop across the diodes in a passive full-wave rectifier. 

Figure \ref{fig:differentialAmpTestBench} shows the Differential Amplifier testbench. 
It generates two triangle waves of equal and opposite amplitude. 
The voltage source simulating the PLL output has an increasing delay imposed on it by the step command. 
The output of the testbench is shown in Figure \ref{fig:differentialAmpTestBenchWaveform}. 
It can be seen that the maximum output voltage increases with as the delay increases. 

A simulation involving more steps was used to produce the plot in Figure \ref{fig:differentialAmpLinearity}. 
The output voltage with 0 delay is 35.3 uV.
The output voltage with the maximum delay of 14.5 ms is 3.31 V.
Figure \ref{fig:differentialAmpLinearity} shows a correlation coefficient of 0.999999959, meaning the differential amplifier easily meets the specification stated in Section \ref{sec:specificationDifferentialAmp}. 

\begin{figure}[H]
    \centering 
    \includegraphics[width=\textwidth]{../Circuits/Images/DifferentialAmp/Schematic}
    \caption{A screencap of the Differential Amplifier Subcircuit}
    \label{fig:differentialAmpSchematic}
\end{figure}

\begin{figure}[H]
    \centering 
    \includegraphics[width=\textwidth]{../Circuits/Images/DifferentialAmp/TestBench}
    \caption{A screencap of the Differential Amplifier Testbench}
    \label{fig:differentialAmpTestBench}
\end{figure}

\begin{figure}[H]
    \centering 
    \includegraphics[width=\textwidth]{../Circuits/Images/DifferentialAmp/TestBenchWaveform}
    \caption{A screencap of the output produced by the Differential Amplifier Testbench}
    \label{fig:differentialAmpTestBenchWaveform}
\end{figure}

\begin{figure}[H]
    \centering 
    \includegraphics[width=\textwidth]{../Circuits/Images/DifferentialAmp/linearityPlot}
    \caption{A plot showing the linearity of the Differential Amplifier}
    \label{fig:differentialAmpLinearity}
\end{figure}

\subsubsection{Digital Logic}
The AD7819 should be used as the ADC.
It provides a parallel interface with with chip select and read data controls. 
The datasheet can be found here\footnote{\url{https://www.analog.com/media/en/technical-documentation/data-sheets/AD7819.pdf}}.

The microcontroller should be interfaced as shown in Figure 4 of the datasheet with the parallel interface interrupted by the 8-bit latch.
Vref should be tied to 3.3 volts through a voltage divider between 5V and GND. 
To perform a conversion, the microcontroller should follow the timing diagram as described in Figure 15 of the datasheet. 

The SN74HC573AN should be used as the 8-bit latch. 
The characteristics of this device far exceed the timing requirements of this circuit in terms of switching speed. 
The latch enable line is active HIGH so an XNOR gate should be used to generate the latch enable signal as specified in Section \ref{sec:systemWideSpecification}.

The XNOR gate should be implemented using the NC7SZ57P6X universal logic gate. 
The datasheet can be found here\footnote{\url{https://www.onsemi.com/pub/Collateral/NC7SZ58-D.PDF}}.
It is a cheap, space-friendly way to produce an XNOR gate.
It should be wired up as shown in Figure 8 of the data sheet, with connections A and B connected to the differentiator outputs and Y connected to the latch enable pin of the 8-bit latch. 
As with the latch, the characteristics of this device exceed the timing requirements of this circuit.
