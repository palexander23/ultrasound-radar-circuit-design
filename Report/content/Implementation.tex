\section{Implementation}
\label{sec:implementation}

\subsection{Modulation}
\subsubsection{Triangle Wave Generator}
The topology implemented for this subcircuit was found in \citetitle{TriangleWaveTopology}\cite{TriangleWaveTopology}. 
It uses an integrator to generate a ramp from the output of a non-inverting schmitt trigger. 
This ramp is fed into the input of the schmitt trigger. When the ramp reaches a schmitt trigger threshold, the polarity of the schmitt trigger output reverses, forcing the integrator to ramp in the opposite direction. 
High precision, rail-to-rail opamps and E96 resistors ensure the amplitude and offset tolerances were met. 

Figure \ref{fig:triangleWaveGeneratorSchematic} shows the schematic, and Figure \ref{fig:triangleWaveGeneratorTestBench} shows the testbench used to test it. 
The Spice error log shows the results of the measurement directives.
The offset is 0.003V and the peak-to-peak voltage is 2.001 V.
The steady-state frequency is measured as 11.3532 Hz.
Each of these falls within the tolerances presented in Section \ref{sec:specificationTriangleWaveGenerator}.
A cutting of the waveform produced can be found in Figure \ref{fig:triangleWaveGeneratorWaveform}. 


\begin{figure}[H]
    \centering 
    \includegraphics[width=\textwidth]{../Circuits/Images/TriangleWaveGenerator/schematic}
    \caption{A screencap of the Triangle Wave Generator Subcircuit}
    \label{fig:triangleWaveGeneratorSchematic}
\end{figure}

\begin{figure}[H]
    \centering 
    \includegraphics[width=\textwidth]{../Circuits/Images/TriangleWaveGenerator/TestBenchScreencap}
    \caption{A screencap of the Triangle Wave Generator Subcircuit}
    \label{fig:triangleWaveGeneratorTestBench}
\end{figure}

\begin{figure}[H]
    \centering 
    \includegraphics[width=\textwidth]{../Circuits/Images/TriangleWaveGenerator/OutputWaveform}
    \caption{A screencap of the Triangle Wave Generator Subcircuit}
    \label{fig:triangleWaveGeneratorWaveform}
\end{figure}