\section{Critical Analysis}
\label{sec:criticalAnalysis}

It was assumed during the completion of this document that the requirement for a bespoke FMCW radar sensor meant that the robot being produced required high levels of performance and had a budget to match.
If not, an off-the-shelf pulse-based device such as the HC-S04\footnote{\url{https://cdn.sparkfun.com/datasheets/Sensors/Proximity/HCSR04.pdf}} would have been used. 
To this end, the system designed is suitable for a moderate to high-budget, high-performance system.
Conversely, this system would not be appropriate for a low-budget, low-performance system such as a toy. 
The components used are high performance and rely on a dual-rail power supply of a kind that would be expensive to implement.  

The component choices themselves work well for the designs in which they are used, but more thought could be put in to optimising for price. 
Only Spice models shipped with LTSpice were used in the simulations but there may be better, more cost-effective devices not found in the LTSpice device library. 
The number of different op-amps could also be reduced. 
Replacing each op-amp in the design with the ADA4000-1 used in the differential amplifier should be investigated as it performed well in the inverting amplifier and the full-wave rectifier.

The lack of knowledge regarding the robot and the ultrasound transducers resulted in some uncertainty with component values.
The ability to set the gain of the Pre-Amp and the inverter in the differential amplifier resulted in reduced bandwidth of both circuits. 
The addition of jumpers to configure the use of the transmission amplifier also adds extra complexity to the system. 
However, it is felt that making the circuit configurable was the right choice for this project. 
Once the system has been built and tested using the real hardware the component values can be set in stone ready for a high volume production run. 
If this was a real project, a test circuit for the PLL could be produced and evaluated separately if development time allowed to ensure the implementation of the support circuitry was correct. 

More research could have been made into PLLs with spice models but a compromise was made with the the chosen device. 
Its low price and simple support circuitry meant that the parameters of the PLL subcircuit would be easy to edit during PCB testing. 
Similarly, the timing requirements of the digital circuits were lenient enough that time was not spent finding spice models for each device so that simulation could be completed. 
In a future design, this digital logic could be replaced with a low-cost CPLD to save on PCB space and board revisions. 

Over all, it is felt that the brief has been met but testing with physical hardware should be performed before a mass-production run of the circuit is advised.
 